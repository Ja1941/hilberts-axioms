\documentclass{article}
\usepackage[utf8]{inputenc}
\usepackage{amsmath}
\usepackage{amsfonts}

\title{format}
\author{}
\date{}

\begin{document}

\newtheorem{theorem}{Theorem}[section]
\newtheorem{lemma}[theorem]{Lemma}
\newtheorem{corollary}[theorem]{Corollary}
\newtheorem{proposition}[theorem]{Proposition}
\newtheorem{definition}[theorem]{Definition}
\newtheorem{remark}[theorem]{Remark}
\numberwithin{equation}{section}

\section{Introduction}

\begin{definition}[Geodesic]
    A geodesic is a curve $\gamma$ that locally minimises the distance. That is, for every point $z_0 \in \gamma$ and $z_1 \in \gamma$ sufficiently close to $z_0$, the segment of $\gamma$ that connects the two points is the shortest curve joining them.
\end{definition}

\begin{remark}
    Geodesics might not minimise the distance between two points as it is defined locally. This is true when geodesics are uniquely determined between the two points.
\end{remark}

\begin{proposition}
    In $\mathbb{H}$, vertical lines are the unique geodesics from $ai$ to $bi$($b>a$). That is, $d_{\mathbb{H}}(ai, bi) = \log{b/a}$.
\end{proposition}

\begin{theorem}[Find geodesics in $\mathbb{H}$]
    For arbitrary two points in $\mathbb{H}$, if they share the same real part, the geodesic is by Prop 1.3 a vertical line. Otherwise, the geodesic is uniquely an arc centred at the real axis.
\end{theorem}

\end{document}
